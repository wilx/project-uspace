\documentclass[paper=B5,DIV=calc,parskip=half]{scrartcl}
\usepackage{ifluatex}

\usepackage{fontspec}
%\defaultfontfeatures{Ligatures={TeX,Common},Scale=MatchLowercase}
\setmainfont[]{Charis SIL}
\setsansfont[Ligatures={TeX,Common}, Scale=MatchLowercase]{TeX Gyre Heros}
\setmonofont[Ligatures=,Scale=MatchLowercase]{DejaVu Sans Mono}

\usepackage{listings}
\usepackage{enumitem}
%\usepackage{enumitem-zref}
\usepackage{polyglossia}
\setmainlanguage[]{english}

\usepackage{uspace}
\usepackage[autostyle]{csquotes}

\author{Václav Haisman\texorpdfstring{%
    \\{\small\href{mailto:vhaisman+uspace@gmail.com?subject=[uspace]}{vhaisman@gmail.com}}}{}}
\date{\today}
\title{\texttt{uspace} user manual}

\PassOptionsToPackage{unicode}{hyperref}
\usepackage{bookmark}
\usepackage{xcolor}
\hypersetup{
  colorlinks,
  linkcolor={red!50!black},
  citecolor={blue!50!black},
  urlcolor={blue!80!black},
  breaklinks=true}
\urlstyle{same}  % don't use monospace font for urls

\usepackage{multicol}
\usepackage{microtype}
\usepackage{metalogo}

\setlength{\emergencystretch}{3em}  % prevent overfull lines

\lstdefinestyle{myLatexStyle}{
  language=[LaTeX]{Tex},
  basicstyle=\ttfamily
}

\lstset{style=myLatexStyle}

\newcommand{\zwsp}{\textsc{zero width space}}
\newcommand{\nnbsp}{\textsc{narrow no-break space}}
\newcommand{\nbsp}{\textsc{non-breaking space}}
\newcommand{\shy}{\textsc{soft hyphen}}
\newcommand{\enquad}{\textsc{en quad}}
\newcommand{\enspaceC}{\textsc{en space}}
\newcommand{\emquad}{\textsc{em quad}}
\newcommand{\emspaceC}{\textsc{em space}}
\newcommand{\threePerEm}{\textsc{three-per-em space}}
\newcommand{\fourPerEm}{\textsc{four-per-em space}}
\newcommand{\sixPerEm}{\textsc{six-per-em space}}
\newcommand{\figuresp}{\textsc{figure space}}
\newcommand{\punctsp}{\textsc{punctuation space}}
\newcommand{\thinsp}{\textsc{thin space}}
\newcommand{\hairsp}{\textsc{hair space}}

\begin{document}
\begin{titlepage}
  \maketitle
\end{titlepage}

\tableofcontents

\section{Introduction}%
%
This \LaTeX{} package gives useful meaning to various Unicode space
characters so that they fulfill their intended function when used in \LaTeX{}
source. It uses \lstinline|\newunicodechar| macro to do it.

Here is a list of the implemented characters
  and their implementations:\nobreak%
\begin{labeling}[\quad]{\nnbsp{}}
\item[\zwsp] \lstinline|\hspace{0pt}|
\item[\nnbsp] \lstinline|\leavevmode\,|
\item[\nbsp{\footnotemark[1]}] \lstinline|~|
\item[\shy{\footnotemark[1]}] \lstinline|\-|
\item[\emquad{\footnotemark[2]}] \lstinline|\quad|
\item[\emspaceC{\footnotemark[2]}] \lstinline|\quad|
\item[\enquad{\footnotemark[3]}] \lstinline|\enskip|
\item[\enspaceC{\footnotemark[3]}] \lstinline|\enskip|
\item[\threePerEm] \lstinline|\hspace{0.3333333333333em}|
\item[\fourPerEm] \lstinline|\hspace{0.25em}|
\item[\sixPerEm] \lstinline|\hspace{0.1666666666666em}|
\item[\figuresp] \lstinline|\leavevmode\hphantom{0}|
\item[\punctsp] \lstinline|\leavevmode\hphantom{.}|
\item[\thinsp] \lstinline|\leavevmode\,\linebreak[0]|
\item[\hairsp] \lstinline|\hspace{0.0833333333333em}|
\footnotetext[1]{This already defined for pdf\LaTeX{} because we use
  \lstinline|inputenc| with \lstinline|utf8| option when compiling with
  pdf\LaTeX{}, therefore this is only defined for \LuaLaTeX{} and
  \XeLaTeX{}.}
\footnotetext[2]{According to Unicode, these two are canonically
  equivalent. See \url{http://unicode.org/notes/tn5/}
  for explanation of the term.}
\footnotetext[3]{These two are also canonically equivalent. See previous
  footnote.}
\end{labeling}

\section{History}%
%
This package would not be what it is without help and comments from people of
\href{http://chat.stackexchange.com/rooms/41/tex-latex-and-friends}{\TeX{},
  \LaTeX{} and Friends} StackExchange chat room and the
\href{http://tex.stackexchange.com/}{\TeX.SE} site itself.

\begin{description}[style=nextline, labelwidth=4.5em, leftmargin=!,
  labelindent=0em]
  \item[\texttt{v0.01}] First published version of this package.
\end{description}

\end{document}
